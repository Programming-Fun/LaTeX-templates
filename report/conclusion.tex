\chapter{Conclusion}

This 6-month internship has been extremely rich and intense. I went from being a very beginner at JavaScript to having a quite good knowledge of it. I made big use of ECMAScript 2015 (or ES6), the last official release of the JavaScript specifications (the standard most JavaScript engines tend to implement). It gave me enough insights to measure efficiently the pros and cons of Object-Oriented Programming compared to Prototype-Based Programming (JavaScript supports both paradigms, although it was historically Prototype-Based).\\
At some point, I did some Functional Programming thanks to JavaScript libraries (and Matthias' willingness), not for a long time, but long enough to inspire me and aroused my curiosity (even though I had already used Haskell in the past). I am pretty sure that I will dig into it any time soon. JavaScript truly is an extremely powerful language, too often misunderstood and disregarded (just like PHP is). It used to be very messy but now, more and more, ECMAScript tries to fix problems and break down misconceptions. JavaScript offers developers a lot of freedom, it is just up to them to keep consistent and avoid common and too well known pitfalls. This freedom is the true strength of JavaScript. Bringing Object-Oriented stuff in the latest specifications is just syntactic sugar. In the meantime, they also added, in ES5 and ES6, features from the functional world, like \lstinline{map}, \lstinline{reduce}, or Promises (although this one is a moot point), which is, in my opinion, really awesome.\\
To briefly conclude about JavaScript, I honestly enjoyed working with this language. It is one of the most powerful languages in the world. It is getting incredibly trendy right now. Everyday there is a new revolutionary framework or library coming out (React and Cycle.js are probably the best examples, not to mention architectures like Facebook's Flux and Relay or the awesome Redux). React programming is the new hotness, thanks to JavaScript. JavaScript's future is, with no doubt, bright.

\bigskip

There is a lot to say about cross-platform development, especially on mobile platforms. I was really doubtful when I first saw the internship offer. I had heard lots of bad things about this, in general. Overall, feedback was pretty bad. So I thought "let's give it a try, do not think about what you heard, take a fresh approach".\\
During the first few weeks, I was somehow amazed by the power of Titanium. It supported most of the features of Android and iOS, and more importantly, it was based on JavaScript. One codebase to rule them all. Then, little by little, came the disillusionment. Obsolete and outdated documentation, bad support of the latest features. I remember when Android Marshmallow was publicly released for general availability (October 2015), we had had to wait a few weeks, solely to be able to compile and run apps on it. Furthermore, there is a growing lack of features, especially those brought by the Google Play Services. I am not confident with iOS so I can not actually tell, but according to my colleagues it is as miserable. Sometimes, we would expect a specific behavior from a piece of code, and it would have a completely different effect from a platform to the other.\\
As I said earlier in this report, cross-platform development, or at least Titanium, is well suited for simple apps with commonly encountered features. There is definitely a comfort zone with Titanium and you do not want to get out of it, at all.\\
Recently, in January 2016, Appcelerator has been acquired by another company. At the Smiths, we saw it as a bad sign given to the community. We foreshadowed a dark future for Titanium. On the opposite world of React-native (based on React), the growing community and the support of Facebook is way more reassuring.\\
For the time being, I am still not convinced at all. Instead of cross-platform solutions, I would rather go with native development (Java or Objective-C/Swift). Anyway, I do not truly believe in mobile apps in the long run...

\bigskip

A third aspect of this internship is obviously the fact that it was a startup. Despite it was my second significant experience in a startup, this one was rather different as The Smiths is an agency. As part of a small team, I was directly exposed to most aspects of the startup life. Besides The Smiths, as I was working in an incubator having dozens of startups, from the big ones employing many people and occupying whole floors to the very small ones (less than five people, sometimes only a founder), I have had the chance to talk to many other people. Some of them would sometimes tell me their stories. I am aware of the Silicon Valley Cult\footnote{\href{https://thinkfaster.co/2015/12/beware-of-the-silicon-valley-cult/}{thinkfaster.co/2015/12/beware-of-the-silicon-valley-cult}} and thus am extremely cautious when dealing with startup offers but I definitely have no regret with that one. The Smiths were unlikely to fail as it was an agency, not a startup selling its own product or service. Being daily among startups was awesome as it taught me a lot. I envision a bright future for many of them. Founding a company some day after graduation has always been on my mind. Now, I have a better understanding of the long way it is.

\bigskip

From a personal point a view, I have bigger dreams. I do not want to work on meaningless projects. I do not believe in the mobile industry. I only see it as entertainment. I am not getting my hopes up about it, I am fairly down-to-Earth; I honestly believe the era of mobile is on its descending slope. I just want to do something that matters -- a bit. Working in an agency for clients means handing them over the project at the end of the contract. There is no point in it for me.\\
Consequently, I will keep on exploring new areas and learning (an engineer, especially a software engineer, never stops learning). Functional programming is something I want to deeply explore. It has been leveraging the entire industry for years (JavaScript is getting more and more functional, Elm is a functional programming language that produces JavaScript, and so on). Most of all, I do not want to be stuck in any particular paradigm nor language. Henceforth, I will try to study many languages, keep an eye on react programming and obviously continue with JavaScript for a bit. Continuous Integration is something I wish was more widely spread, it is definitely a best practice in the \nth{21} century. Testing as well, should be more prominent. Computer Science is still very recent and yet already old. Things are going insanely fast and it is almost always impossible to catch up. I am constantly reshaping what I really want to do -- it might never end -- but it is getting more and more clear to me.